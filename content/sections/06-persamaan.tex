
\section{\MakeUppercase{Penulisan Persamaan}}
Persamaan secara berurutan diikuti dengan penomoran angka dalam tanda kurung dengan margin rata kanan, seperti dalam (\ref{eq:example}). Gunakan equation editor untuk membuat persamaan. Beri spasi tab dan tulis nomor persamaan dalam tanda kurung. Untuk membuat persamaan Anda lebih rapat, gunakan tanda garis miring ( / ), fungsi pangkat, atau pangkat yang tepat. Gunakan tanda kurung untuk menghindari kerancuan dalam pemberian angka pecahan. Jelaskan persamaan saat berada dalam bagian dari kalimat, seperti berikut

\begin{equation}
\int_0^{r^2} F(r,\phi) \, dr \, d\phi = [\sigma r^2/(2\mu_0)] \cdot \int_0^{\infty} \exp(-\lambda|z_j - z_i|) \lambda^{-1} J_1(\lambda r^2) J_0(\lambda r_i) \, d\lambda
\label{eq:example}
\end{equation}

Pastikan bahwa simbol-simbol di dalam persamaan telah didefinisikan sebelum persamaan atau langsung mengikuti setelah persamaan muncul. Simbol diketik dengan huruf miring ($T$ mengacu pada suhu, tetapi T merupakan satuan Tesla). Mengacu pada "(\ref{eq:example})", bukan "Pers. (\ref{eq:example})" atau "persamaan (\ref{eq:example})", kecuali pada awal kalimat: "Persamaan (\ref{eq:example}) merupakan ...".


