
\section{\MakeUppercase{Style Halaman}}
Paragraf harus teratur. Semua paragraf harus rata, yaitu sama-sama rata kiri dan dan rata kanan.

\subsection{Huruf-huruf Dokumen}
Seluruh dokumen harus dalam Times New Roman atau Times font. Font tipe 3 tidak boleh digunakan. Jenis font lain dapat digunakan jika diperlukan untuk tujuan khusus.

Fitur ukuran font dapat dilihat pada Tabel \ref{tab:font_sizes}.

\subsection{Judul dan Penulis}
Judul harus dalam font biasa berukuran 20 pt. Nama pengarang harus dalam font biasa berukuran 11 pt. Jumlah kata judul maksimal 12 kata.

Judul dan pengarang harus dalam format kolom tunggal dan harus terpusat. Setiap awal kata dalam judul harus huruf besar, kecuali untuk kata-kata pendek seperti, "sebuah", "dan", "di", "oleh", "untuk", "dari", "pada", "atau", dan sejenisnya. Penulisan penulis tidak boleh menunjukkan nama jabatan (misalnya Dosen Pembimbing), gelar akademik (misalnya Dr) atau keanggotaan dari setiap organisasi profesional (misalnya Senior Member IEEE).

Agar tidak membingungkan, jika ada nama keluarga maka ditulis di bagian terakhir dari masing-masing nama pengarang (misalnya Agus AK Sumitro). Setiap afiliasi harus dirmasukkan, setidaknya, nama perusahaan dan nama negara tempat penulis (misalnya Prime Education Centre Pty Ltd, INDONESIA). Alamat email dwajibkan bagi penulis yang bersangkutan.

\subsection{Bagian Heading}
Sebaiknya tidak lebih dari 3 tingkat untuk heading. Semua heading harus dalam font 10pt. Setiap kata dalam suatu heading harus berhuruf besar, kecuali untuk kata-kata pendek seperti yang tercantum dalam Bagian III-B.

\subsubsection{Heading Level1} Heading level 1 harus dalam Small Caps, terletak di tengah-tengah dan menggunakan penomoran angka Romawi huruf besar. Sebagai contoh, lihat heading "III. Style Halaman" dari dokumen ini. Heading level 1 yang tidak boleh menggunakan penomoran adalah "Ucapan Terima Kasih" dan "Referensi".

\subsubsection{Heading Level-2} Heading level 2 harus miring (italic), merapat ke kiri dan dinomori menggunakan abjad huruf besar. Sebagai contoh, lihat heading "C. Bagian heading" di atas.

\subsubsection{Heading Level-3} Heading level-3 harus diberi spasi, miring, dan dinomori dengan angka Arab diikuti dengan tanda kurung kanan. Heading level 3 harus diakhiri dengan titik dua. Isi dari bagian level 3 bersambung mengikuti judul heading dengan paragraf yang sama. Sebagai contoh, bagian ini diawali dengan heading level 3.


