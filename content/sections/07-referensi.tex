
\section{\MakeUppercase{Referensi}}
Judul pada bagian Referensi tidak boleh bernomor. Semua item referensi berukuran font 8 pt. Silakan gunakan gaya tulisan miring dan biasa untuk membedakan berbagai perbedaan dasar seperti yang ditunjukkan pada bagian Referensi. Penomoran item referensi diketik berurutan dalam tanda kurung siku (misalnya [1]).

Ketika Anda mengacu pada item referensi, silakan menggunakan nomor referensi saja, misalnya \cite{metev1998}. Jangan menggunakan "Ref. \cite{breckling1989}" atau "Referensi \cite{zhang1999}", kecuali pada awal kalimat, misalnya "Referensi \cite{zhang1999} menunjukkan bahwa ...". Dalam penggunaan beberapa referensi masing-masing nomor diketik dengan kurung terpisah (misalnya \cite{metev1998}, \cite{breckling1989}, \cite{zhang1999}--\cite{shell2002}). Beberapa contoh item referensi dengan kategori yang berbeda ditampilkan pada bagian Referensi yang meliputi:
\begin{itemize}
\item contoh buku pada \cite{metev1998}
\item contoh seri buku dalam \cite{breckling1989}
\item contoh artikel jurnal di \cite{zhang1999}
\item contoh paper seminar di \cite{wegmuller2000}
\item contoh paten dalam \cite{sorace1997}
\item contoh website di \cite{ieee2002}
\item contoh dari suatu halaman web di \cite{shell2002}
\item contoh manual databook dalam \cite{flexchip1996}
\item contoh datasheet dalam \cite{pdca1999}
\item contoh tesis master di \cite{karnik1999}
\item contoh laporan teknis dalam \cite{padhye1999}
\item contoh standar dalam \cite{ieee8021997}
\end{itemize}


